%-----------------------------------------------------------------------------
%
%               Template for sigplanconf LaTeX Class
%
% Name:         sigplanconf-template.tex
%
% Purpose:      A template for sigplanconf.cls, which is a LaTeX 2e class
%               file for SIGPLAN conference proceedings.
%
% Guide:        Refer to "Author's Guide to the ACM SIGPLAN Class,"
%               sigplanconf-guide.pdf
%
% Author:       Paul C. Anagnostopoulos
%               Windfall Software
%               978 371-2316
%               paul@windfall.com
%
% Created:      15 February 2005
%
%-----------------------------------------------------------------------------


\documentclass[nocopyrightspace]{sigplanconf}

% The following \documentclass options may be useful:

% preprint      Remove this option only once the paper is in final form.
% 10pt          To set in 10-point type instead of 9-point.
% 11pt          To set in 11-point type instead of 9-point.
% authoryear    To obtain author/year citation style instead of numeric.

\usepackage{amsmath}


\begin{document}

\special{papersize=8.5in,11in}
\setlength{\pdfpageheight}{\paperheight}
\setlength{\pdfpagewidth}{\paperwidth}

\conferenceinfo{CONF 'yy}{Month d--d, 20yy, City, ST, Country} 
\copyrightyear{20yy} 
\copyrightdata{978-1-nnnn-nnnn-n/yy/mm} 
\doi{nnnnnnn.nnnnnnn}

% Uncomment one of the following two, if you are not going for the 
% traditional copyright transfer agreement.

%\exclusivelicense                % ACM gets exclusive license to publish, 
                                  % you retain copyright

%\permissiontopublish             % ACM gets nonexclusive license to publish
                                  % (paid open-access papers, 
                                  % short abstracts)

\titlebanner{banner above paper title}        % These are ignored unless
\preprintfooter{short description of paper}   % 'preprint' option specified.

\title{Proposal: Certified LLVM}
% \subtitle{Proposal for my PhD}

\authorinfo{Jeehoon Kang}
           {Software Foundations Laboratory}
           {}
           % {jhkang@ropas.snu.ac.kr}

\maketitle

\begin{abstract}
  I aim to certify the LLVM compiler framework, a practical foundation
  for real-world programming language tools such as compilers and
  static/dynamic analyzers.  Previous works on compiler verification
  typically focused on building new compilers.  These new compilers
  were used for safety-critical systems such as airplanes.  On the
  other hand, I target on LLVM to certify a general-purpose compiler
  that is daily used for all softwares.

  More specifically, I would like to \emph{(i)} formalize the LLVM
  syntax and semantics as is used in the real-world, \emph{(ii)}
  establish a powerful reasoning principle for LLVM, and \emph{(iii)}
  prove optimization level 1 (\texttt{-O1}) for the LLVM compiler
  based on the reasoning principle.

  I expect lots of theoretical and practical challenges for this
  project including \emph{(i)} formalizing standards consisting of
  ambiguous prose, \emph{(ii)} reducing verification efforts by
  re-using previous developments, \emph{(iii)} dealing with relaxed
  behavior of memories, and \emph{(iv)} supporting compositional
  reasoning principle essential in modular development.

  I would like to do this project for my PhD.  The project will be a
  monumental work, so I anticipate \emph{your} participation, along
  with that from all around the world.
\end{abstract}

\section{Context}
Certification of real-world compiler is a long-standing goal.
Compiler is an indispensable component of modern computer systems.
However, real-world compilers like GCC and LLVM are known to contain
lots of bugs \cite{Yang:2011:FUB:1993498.1993532,
  Chen:2013:TCF:2491956.2462173}, even in 2010s after 50+ years of
compiler construction.  Compiler bugs can be detrimental to software
reliability since the bugs nullify analysis and reasoning on source
program.  Currently, certification is the only option to ensure the
absence of compiler bugs \cite{Yang:2011:FUB:1993498.1993532}.

\paragraph{Previous Work}
There are lots of outstanding works on compiler certification.  In the
practical side, most notably, CompCert \cite{Leroy-Compcert-CACM} is
the first certified compiler for a significant subset of C.
CompCertTSO \cite{Sevcik:2013:CVC:2487241.2487248} is a variant of
CompCert to deal with the TSO \emph{relaxed memory model}
(\S\ref{sec:relaxed}) of the x86 and Sparc architectures.  Vellvm
\cite{Zhao:2012:FLI:2103656.2103709} is a certified LLVM framework
upon which \texttt{mem2reg} \cite{Zhao:2013:FVS:2491956.2462164} and
\texttt{instcombine} [our submitted paper] passes are verified.
Vellvm is the first project that aims to certify a real-world
compiler.  On the other hand, in the theoretical side, Hur et
al. conducted a line of research of \emph{compositional reasoning}
(\S\ref{sec:compose}) for modular proof of compiler certification
\cite{Hur:2012:MBK:2103656.2103666}.

\subsection{Problem}
However, I see a need for future works to certify a real-world
compiler for the following reasons:

\paragraph{Limited Tool Support} 
CompCert and CompCertTSO is built for verification and they support
relatively limited set of tools.  Particularly, CompCert's
\texttt{ccomp} is not an out-of-box replacement of \texttt{gcc}.  To
this end, it is needed to verify an existing, widely used compiler
such as GCC and LLVM.  Vellvm is the only project in this approach.

\paragraph{Unfaithful Semantics}
existing certified compilers do not faithfully model the semantics of
programming languages and architectures.  For concurrency, CompCert's
and Vellvm's semantics deals with only single-thread programs,
entirely ignoring concurrency.  CompCertTSO addresses concurrency and
deals with the TSO memory model, but C11/C++11 semantics of
concurrency is not entirely captured in the TSO model.  Moreover, even
except for concurrency, Vellvm's semantics is not entirely faithful to
the official LLVM semantics.  For example, Vellvm did not formalize
vector types (statically sized arrays), indirect jumps, and
exceptions.

\paragraph{High Verification Cost}
Certified compiler projects took several person-years to develop.  A
real-world compiler will be a bigger software than previously
certified ones.  Thus its certification will take more efforts, if
verification cost is not properly managed.

\paragraph{Low Performance of Compilation Result}
In a benchmark I performed, CompCert's \texttt{ccomp} is 20\% slower
than LLVM's \texttt{clang -O1} (geometric mean).  CompCert's
performance may be acceptable for some use cases, but is
unsatisfactory for a general-purpose compiler.  To certify a
real-world compiler, the performance issue should be addressed.

\section{Goal}

In my PhD, I would like to address the above-mentioned problems in
certification of a real-world compiler.  As an demonstration, I will
certify the LLVM compiler framework.  Specifically, I would like to do
the followings:

\paragraph{Faithfully Formalizing of LLVM}
I would like to formalize the syntax and semantics of the LLVM IR as
faithful as the production-ready level.  I would like to formalize
program constructs missing in Vellvm, such as vector types, and
formalize LLVM's concurrency and relaxed memory model.

\paragraph{Designing Powerful Reasoning Principle}
To reduce verification cost, it is essential to design a powerful
reasoning principle for compiler correctness.  By powerful we expect
two aspects:
\begin{itemize}
\item Applicable to many situations (optimization cases).  It would be
  too costly to design a proof method for each optimization case.
  Thus reasoning principle should be as general as possible.
\item Easy to separate out irrelevant (``framing'') details and focus
  on real tasks.  For example, if only two variables \texttt{x} and
  \texttt{y} is relevant to an optimization, we would like to separate
  out other variables from our concern and focus on the two variables
  in reasoning.
\end{itemize}
These aspects will enable a modular development of verification, thus
significantly decreasing cost.

\paragraph{Certifying Optimization Level 1}
On the formalized LLVM IR, we would like to verify optimization level
1 (\texttt{-O1}) with the designed reasoning principle.  Optimization
level 1 will ensure high performance of compilation result.  For
verification cost, fortunately, out of 43 optimization passes in
\texttt{-O1}, 27 seems to be possible to prove with Hoare-style
reasoning (relatively basic reasoning principle used in our recently
submitted paper).  Thus I expect it would take person-years of efforts
to verify \texttt{-O1}, relatively cheap compared to the size of the
software.

\section{Challenge}
Towards the goal, I expect lots of theoretical and practical
interesting challenges.  I am eager to solve the following challenges:

\paragraph{Formalization of Real-World Languages}
The official LLVM documentations are written without intention to
clearly define the language, so many corner cases are subject to
interpretation.

\paragraph{Re-use of Previous Development}
Compiler certification is not a fresh problem, and lots of certified
compilers are developed, including CompCert, CompCertTSO, and Vellvm.
Each project took person-years of efforts to develop, so it is an
unacceptable waste to reinvent the entire wheel in this project.  Thus
I would like to re-use previous developments as much as possible,
while leaving out unsatisfactory aspects of them.

\paragraph{Memory Model}
CompCertTSO is based on the TSO memory model
\cite{Sewell:2010:XRU:1785414.1785443}, a relaxed memory model of x86
and Sparc architectures.  We may restrict our concern to the TSO
model, instead of more relaxed models such as C11/C++11 and ARM
models, to re-use CompCertTSO's memory model as-is.

\paragraph{Relaxed Memory Model}\label{sec:relaxed}
In the context of semantics engineering for single-thread programs,
memory has long been abstracted to map from locations to values.  Many
works on compiler certification and static analysis relies on this
simple memory abstraction.  This abstraction is useful and arguably
correct, at least for single-thread programs.  However, the
abstraction is broken for multi-thread programs.  C11/C++11 standards
clearly declare that memory is not just a map
\cite{Batty:2011:MCC:1926385.1926394}; similarly to all modern
architectures including x86 \cite{Sewell:2010:XRU:1785414.1785443} and
ARM \cite{Sarkar:2011:UPM:1993498.1993520}.

Here is an example of the \emph{relaxed} memory behavior, which is not
explained in the simple model:
\begin{verbatim}
int data = 0, flag = 0;
{{{ data = 1; flag = 1;
||| while (!flag) {}
    assert (data==1);
}}}
\end{verbatim}
Here, \texttt{\{\{\{$p_1$|||$p_2$\}\}\}} is a thread-wise composition
of two programs $p_1$ and $p_2$.  In the simple model, the assertion
should succeed, since $\texttt{flag}=1$ means $\texttt{data}=1$.
However, according to the C11 standard, the assertion may fail due to
relaxed memory behavior.  The reason is that the write of
\texttt{data} at the first thread is not particularly synchronized
with the read of \texttt{data} at the second thread.  I defer the
details on this example and the relaxed memory model to a tutorial in
the ROSAEC Workshop.

In the era of multi- and many-core systems, certification of
real-world compiler should address the relaxed memory behavior.
However, LLVM does not clearly declare its memory model, so we have to
define a model.  Furthermore, reasoning on the relaxed memory model
seems to be much more complicated than on the simple model
\cite{Aaron:GPS}.  Design of a manageable relaxed memory model and a
powerful, easy-to-use reasoning principle on the model would be
essential in certification of LLVM.

\paragraph{Compositional Reasoning}\label{sec:compose}

For modular development and proof of certified software, it is
essential to prove in part, then \emph{composite} the partial proof to
make the whole proof.  In the context of compiler certification, it is
particularly useful to composite \emph{simulation} relations ($\sim$),
both vertically and horizontally:
\begin{itemize}
\item Vertical composition: suppose there is compilation $C_1: p_0
  \mapsto p_1$ and $C_2: p_1 \mapsto p_2$, where $p_i$ are program
  fragments (such as functions).  Horizontal composition ($C_2 \circ
  C_1$) respects the simulation relation, if $p_0 \sim p_1$ and $p_1
  \sim p_2$ implies $p_0 \sim p_2$.
\item Horizontal composition: suppose there is compilation $C_p: p_1
  \mapsto p_2$ and $C_q: q_1 \mapsto q_2$, where $p_i$ and $q_i$ are
  program fragments.  Suppose $p \star q$ is the juxtaposition of
  program fragments $p$ and $q$.  Horizontal composition ($C_p \star
  C_q: p_1 \star q_1 \mapsto p_2 \star q_2$) respects the simulation
  relation, if $p_1 \sim p_2$ and $q_1 \sim q_2$ implies $p_1 \star
  q_1 \sim p_2 \star q_2$.
\end{itemize}

I am lucky in that Prof. Hur is a pioneer in compositional reasoning.
He has developed a powerful reasoning principle
\cite{Hur:2012:MBK:2103656.2103666}, but design of a principle for
concurrent, relaxed memory model is yet an open question.  I would
like to solve this problem during the development of certified LLVM.

\section{Recruit}
This project will be a monumental work, so I would like to co-work
with many people.  To this end, Prof. Hur contacted Prof. Vafeiadis in
MPI-SWS (expert in C/C++, relaxed memory model and compiler
verification) and Prof. Zdancewic in UPenn (author of Vellvm).  I am
currently mentoring two interns, hoping that the two join the project
in the future.  Also, most importantly, I anticipate \emph{your}
participation if you are interested in this project.

% We recommend abbrvnat bibliography style.

\bibliographystyle{abbrvnat}

% The bibliography should be embedded for final submission.

\bibliography{references}

\end{document}

%                       Revision History
%                       -------- -------
%  Date         Person  Ver.    Change
%  ----         ------  ----    ------

%  2013.06.29   TU      0.1--4  comments on permission/copyright notices

