%-----------------------------------------------------------------------------
%
%               Template for sigplanconf LaTeX Class
%
% Name:         sigplanconf-template.tex
%
% Purpose:      A template for sigplanconf.cls, which is a LaTeX 2e class
%               file for SIGPLAN conference proceedings.
%
% Guide:        Refer to "Author's Guide to the ACM SIGPLAN Class,"
%               sigplanconf-guide.pdf
%
% Author:       Paul C. Anagnostopoulos
%               Windfall Software
%               978 371-2316
%               paul@windfall.com
%
% Created:      15 February 2005
%
%-----------------------------------------------------------------------------


\documentclass[nocopyrightspace]{sigplanconf}

% The following \documentclass options may be useful:

% preprint      Remove this option only once the paper is in final form.
% 10pt          To set in 10-point type instead of 9-point.
% 11pt          To set in 11-point type instead of 9-point.
% authoryear    To obtain author/year citation style instead of numeric.

\usepackage{amsmath}


\begin{document}

\special{papersize=8.5in,11in}
\setlength{\pdfpageheight}{\paperheight}
\setlength{\pdfpagewidth}{\paperwidth}

\conferenceinfo{CONF 'yy}{Month d--d, 20yy, City, ST, Country} 
\copyrightyear{20yy} 
\copyrightdata{978-1-nnnn-nnnn-n/yy/mm} 
\doi{nnnnnnn.nnnnnnn}

% Uncomment one of the following two, if you are not going for the 
% traditional copyright transfer agreement.

%\exclusivelicense                % ACM gets exclusive license to publish, 
                                  % you retain copyright

%\permissiontopublish             % ACM gets nonexclusive license to publish
                                  % (paid open-access papers, 
                                  % short abstracts)

\titlebanner{banner above paper title}        % These are ignored unless
\preprintfooter{short description of paper}   % 'preprint' option specified.

\title{Certified LLVM}
\subtitle{Proposal for my PhD}

\authorinfo{Jeehoon Kang}
           {Software Foundations Laboratory}
           {jhkang@ropas.snu.ac.kr}

\maketitle

\begin{abstract}
  I aim to \emph{certify} the LLVM compiler framework, a practical
  foundation for real-world programming language tools such as
  compilers and static/dynamic analyzers.  Previous works on compiler
  verification typically focused on building new compilers.  These new
  compilers were used for safety-critical systems such as airplanes.
  On the other hand, I target on LLVM to certify a compiler that is
  daily used for all softwares.

  More specifically, I would like to \emph{(i)} formalize the LLVM
  syntax and semantics as is used in the real-world, \emph{(ii)}
  establish a powerful reasoning principle for LLVM, and \emph{(iii)}
  prove optimization level 1 (\texttt{-O1}) for the LLVM compiler
  based on the reasoning principle.

  I expect a lot of theoretical/practical challenges for this project
  including \emph{(i)} formalizing standards consisting of ambiguous
  prose, \emph{(ii)} reducing verification efforts by re-using
  previous developments, \emph{(iii)} dealing with relaxed behavior of
  memories, and \emph{(iv)} supporting compositional reasoning
  principle essential in modular development.

  I would like to do this project for my PhD.  The project will be a
  monumental work, so I anticipate \emph{your} participation, along
  with that from all around the world.
\end{abstract}

\section{Context}
Certification of real-world compiler is a long-standing goal.
Compiler is an indispensable component of modern computer systems.
However, real-world compilers like GCC and LLVM are known to contain
lots of bugs \cite{TODO}, even in 2010s after 50+ years of compiler
construction.  Compiler bugs can be detrimental to software
reliability since the bugs nullify analysis and reasoning on source
program.  Currently, certification is the only option to ensure the
absence of compiler bugs \cite{TODO}.

\paragraph{Previous Work}
There are lots of outstanding works on compiler certification.  In the
practical side, most notably, CompCert \cite{TODO} is the first
certified compiler for a significant subset of C.  CompCertTSO
\cite{TODO} is a variant of CompCert to deal with the TSO
\emph{relaxed memory model} (\S\ref{sec:relaxed}) of the x86 and Sparc
architectures.  Vellvm \cite{TODO} is a certified LLVM framework upon
which \texttt{mem2reg} \cite{TODO} and \texttt{instcombine}
\cite{TODO} passes are verified.  Vellvm is the first project to
certify a real-world compiler.  On the other hand, in the theoretical
side, Hur et al. conducted a line of research of \emph{compositional
  reasoning} (\S\ref{sec:compose}) for modular proof of compiler
certification \cite{TODO}.  Chlipala built a certified compiler for a
ML-like language, focusing on proof automation \cite{TODO}.

\paragraph{Need for Future Works}
However, I see a need for future works on compiler certification for
the following reasons:
\begin{itemize}
\item Certification of \emph{actually} real-world compiler: to certify
  a real-world compiler, it is essential to faithfully certify an
  existing, widespreadly used compiler.  Unfortunately, CompCert and
  CompCertTSO is built for verification and not widely used.
  Furthermore, CompCert's and Vellvm's semantics deals with only
  single-thread programs, entirely ignoring concurrency.  CompCertTSO
  addresses concurrency and deals with the TSO memory model, but
  C11/C++11 semantics of concurrency is not entirely captured in the
  TSO model.  Moreover, even except for concurrency, Vellvm's
  semantics is not entirely faithful to the official LLVM semantics.
  For example, Vellvm did not formalize vector types (statically sized
  arrays), indirect jumps, and exceptions.  

  I would like to address above issues and formalize the syntax and
  semantics of the LLVM IR as faithful as the production-ready level.

\item Powerful peasoning principle: TODO

\item Low verification cost: TODO

\end{itemize}

\section{Goal}
TODO

\paragraph{Formalization of LLVM}
TODO

\paragraph{Powerful Reasoning Principle}
TODO

\paragraph{Certification of Optimization Level 1}
TODO

\section{Challenge}
TODO

\subsection{Formalization of Real-World Languages and Machines}

\subsection{Re-use of Previous Development}
Compiler certification is not a fresh problem, and lots of certified
compilers are developed, including CompCert \cite{TODO}, CompCertTSO
\cite{TODO}, and Vellvm \cite{TODO}.  Each project took person-years
of efforts to develop, so it is an unacceptable waste to reinvent the
entire wheel in this project.  Thus I would like to re-use previous
developments as much as possible, while leaving out unsatisfactory
aspects of them.

\paragraph{Memory Model}
CompCertTSO is based on the TSO memory model \cite{TODO}, a relaxed
memory model of x86 and Sparc architectures.  We may restrict our
concern to the TSO model, instead of more relaxed models such as
C11/C++11 and ARM models, to re-use CompCertTSO's memory model as-is.

\paragraph{Semantics}

\subsection{Relaxed Memory Model}\label{sec:relaxed}
In the context of semantics engineering for single-thread programs,
memory has long been abstracted to map from locations to values.  Many
works on compiler certification and static analysis relies on this
simple memory abstraction.  This abstraction is useful and arguably
correct, at least for single-thread programs.  However, the
abstraction is broken for multi-thread programs.  C11/C++11 standards
clearly declare that memory is not just a map \cite{TODO}; similarly
to all modern architectures including x86 \cite{TODO} and ARM
\cite{TODO}.

Here is an example of the \emph{relaxed} memory behavior, which is not
explained in the simple model:
\begin{verbatim}
int data = 0, flag = 0;
{{{ data = 1; flag = 1;
||| while (!flag) {}
    assert (data==1);
}}}
\end{verbatim}
Here, \texttt{\{\{\{$p_1$|||$p_2$\}\}\}} is a thread-wise composition
of two programs $p_1$ and $p_2$.  In the simple model, the assertion
should succeed, since $\texttt{flag}=1$ means $\texttt{data}=1$.
However, according to the C11 standard, the assertion may fail due to
relaxed memory behavior \cite{TODO}.  The reason is that the write of
\texttt{data} at the first thread is not particularly synchronized
with the read of \texttt{data} at the second thread.  I defer the
details on this example and the relaxed memory model to a tutorial in
the ROSAEC Workshop.

In the era of multi- and many-core systems, certification of
real-world compiler should address the relaxed memory behavior.
However, LLVM does not clearly declare its memory model, so we have to
define a model.  Furthermore, reasoning on the relaxed memory model
seems to be much more complicated than on the simple model
\cite{TODO}.  Design of a manageable relaxed memory model and a
powerful, easy-to-use reasoning principle on the model would be
essential in certification of LLVM.

\subsection{Compositional Reasoning}\label{sec:compose}

For modular development and proof of certified software, it is
essential to prove in part, then \emph{composite} the partial proof to
make the whole proof.  In the context of compiler certification, it is
particularly useful to composite \emph{simulation} relations ($\sim$),
both vertically and horizontally:
\begin{itemize}
\item Vertical composition: suppose there is compilation $C_1: p_0
  \mapsto p_1$ and $C_2: p_1 \mapsto p_2$, where $p_i$ are program
  fragments (such as functions).  Horizontal composition ($C_2 \circ
  C_1$) respects the simulation relation, if $p_0 \sim p_1$ and $p_1
  \sim p_2$ implies $p_0 \sim p_2$.
\item Horizontal composition: suppose there is compilation $C_p: p_1
  \mapsto p_2$ and $C_q: q_1 \mapsto q_2$, where $p_i$ and $q_i$ are
  program fragments.  Suppose $p \star q$ is the juxtaposition of
  program fragments $p$ and $q$.  Horizontal composition ($C_p \star
  C_q: p_1 \star q_1 \mapsto p_2 \star q_2$) respects the simulation
  relation, if $p_1 \sim p_2$ and $q_1 \sim q_2$ implies $p_1 \star
  q_1 \sim p_2 \star q_2$.
\end{itemize}

I am lucky in that Prof. Hur is a pioneer in compositional reasoning.
He has developed a powerful reasoning principle \cite{TODO}, but
design of a principle for concurrent, relaxed memory model is yet an
open question.  I would like to solve this problem during the
development of certified LLVM.

\section{Plan}

\paragraph{Team Building}
This project will be a monumental work, so I would like to co-work
with many people.  To this end, Prof. Hur contacted Prof. Vafeiadis in
MPI-SWS (expert in C/C++, relaxed memory model and compiler
verification) and Prof. Zdancewic in UPenn (author of Vellvm).  I am
currently mentoring two interns, hoping that the two join the project
in the future.  Also, I anticipate \emph{your} participation if you
are interested in this project.

\paragraph{Verification of Optimization Level 1}
TODO

\paragraph{Designing Reasoning Principle}
TODO

% We recommend abbrvnat bibliography style.

\bibliographystyle{abbrvnat}

% The bibliography should be embedded for final submission.

\bibliography{references}

\end{document}

%                       Revision History
%                       -------- -------
%  Date         Person  Ver.    Change
%  ----         ------  ----    ------

%  2013.06.29   TU      0.1--4  comments on permission/copyright notices

