%-----------------------------------------------------------------------------
%
%               Template for sigplanconf LaTeX Class
%
% Name:         sigplanconf-template.tex
%
% Purpose:      A template for sigplanconf.cls, which is a LaTeX 2e class
%               file for SIGPLAN conference proceedings.
%
% Guide:        Refer to "Author's Guide to the ACM SIGPLAN Class,"
%               sigplanconf-guide.pdf
%
% Author:       Paul C. Anagnostopoulos
%               Windfall Software
%               978 371-2316
%               paul@windfall.com
%
% Created:      15 February 2005
%
%-----------------------------------------------------------------------------


\documentclass[nocopyrightspace]{sigplanconf}

% The following \documentclass options may be useful:

% preprint      Remove this option only once the paper is in final form.
% 10pt          To set in 10-point type instead of 9-point.
% 11pt          To set in 11-point type instead of 9-point.
% authoryear    To obtain author/year citation style instead of numeric.

\usepackage{amsmath}


\begin{document}

\special{papersize=8.5in,11in}
\setlength{\pdfpageheight}{\paperheight}
\setlength{\pdfpagewidth}{\paperwidth}

\conferenceinfo{CONF 'yy}{Month d--d, 20yy, City, ST, Country} 
\copyrightyear{20yy} 
\copyrightdata{978-1-nnnn-nnnn-n/yy/mm} 
\doi{nnnnnnn.nnnnnnn}

% Uncomment one of the following two, if you are not going for the 
% traditional copyright transfer agreement.

%\exclusivelicense                % ACM gets exclusive license to publish, 
                                  % you retain copyright

%\permissiontopublish             % ACM gets nonexclusive license to publish
                                  % (paid open-access papers, 
                                  % short abstracts)

\titlebanner{banner above paper title}        % These are ignored unless
\preprintfooter{short description of paper}   % 'preprint' option specified.

\title{Certified LLVM}
\subtitle{Proposal for my PhD}

\authorinfo{Jeehoon Kang}
           {Software Foundations Laboratory}
           {jhkang@ropas.snu.ac.kr}

\maketitle

\begin{abstract}
  I aim to \emph{certify} the LLVM compiler framework, a practical
  foundation for real-world programming language tools such as
  compilers and static/dynamic analyzers.  TODO: why this work is
  important?

  More specifically, I would like to \emph{(i)} formalize the LLVM
  syntax and semantics as is used in the real-world, \emph{(ii)}
  establish a powerful reasoning principle for LLVM, and \emph{(iii)}
  prove level 1 optimization (\texttt{-O1}) for the LLVM compiler
  based on the reasoning principle.

  I expect a lot of theoretical/practical challenges for this project
  including \emph{(i)} formalizing standards consisting of ambiguous
  prose, \emph{(ii)} reducing efforts by re-using previous
  developments, \emph{(iii)} dealing with relaxed behavior of
  memories, and \emph{(iv)} supporting compositional reasoning
  principle essential in modular development.

  I would like to lead this project for my PhD degree.  The project
  will be a monumental work, so I anticipate \emph{your}
  participation, along with that from all around the world.
\end{abstract}

\section{Context}
TODO

\paragraph{Problem}
TODO(real-world)

\paragraph{Previous Work}
TODO

\section{Goal}
TODO

\section{Challenge}
TODO

\subsection{Formalization of Real-World Languages and Machines}
\subsection{Re-use of Previous Development}
\subsection{Relaxed Memory Model}
\subsection{Compositional Reasoning}

\section{Plan}
TODO

% We recommend abbrvnat bibliography style.

\bibliographystyle{abbrvnat}

% The bibliography should be embedded for final submission.

\bibliography{references}

\end{document}

%                       Revision History
%                       -------- -------
%  Date         Person  Ver.    Change
%  ----         ------  ----    ------

%  2013.06.29   TU      0.1--4  comments on permission/copyright notices

